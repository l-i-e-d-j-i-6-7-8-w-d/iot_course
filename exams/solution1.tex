% !TeX root = ./kernel/EasySolution.tex

\section*{Solution du test de connaissances \MarksOne}
\begin{enumerate}
    \item Définition des termes:
          \begin{itemize}
              \item{\textbf{Les réseaux intelligents} sont des réseaux matériels de distributions de fluides (électricité, eau, gaz, pétrole...), \emph{et/ou d'information (télécommunications)} qui ont été « augmentés » (rendus intelligents) par des systèmes informatiques, capteurs, interfaces informatiques et électromécaniques leur donnant des capacités d'échange bidirectionnel et parfois une certaine capacité d'autonomie en matières de calcul et gestion de flux et traitement d'information.}
              \item \textbf{Capteurs, signal, échantillonnage et actionneurs}
                    \begin{itemize}
                        \item Un capteur est un dispositif transformant l'état d'une grandeur physique observée en une grandeur utilisable appelé signal.
                        \item Un signal est formellement définit comme une fonction d'une ou plusieurs variables qui transmettent des informations sur la nature d'un phénomène physique.
                        \item L'échantillonnage est la première étape dans la numérisation d'un signal et consiste à convertir le signal du message en une séquence de nombres, chaque nombre représentant l'amplitude du signal du message à un instant discrét donné.
                        \item Actionneurs: dispositifs physiques qui déclenchent une action précise lorsqu'ils reçoivent un signal externe.
                    \end{itemize}
          \end{itemize}
    \item Donner la signification des protocoles d'échange de données, de et de sécurité des réseaux intélligents suivants:
          \begin{enumerate}
              \item HTTP:  Hypertext Transfer Protocol
              \item M2M : Machine to machine communication Protocol
              \item CoAP: Constrained Application Protocol
              \item MQTT: Message Queuing Telemetry Transport
              \item SSL: Secure Socket Layer
              \item TLS: Transport Layer Security
              \item DTLS: Datagram Transport Layer Security
          \end{enumerate}
    \item Donner et définissez deux autres protocoles des réseaux intelligents de votre choix.
          \begin{itemize}
              \item DDS: Data-Distribution Service
              \item AMQP : Advanced Message Queuing Protocol
              \item XMPP: Extensible Messaging and Presence Protocol
          \end{itemize}
\end{enumerate}

